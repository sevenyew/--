\documentclass[UTF-8]{ctexart} 
\usepackage[a4paper,left=2cm,right=2cm,top=2.5cm,bottom=2.5cm]{geometry}
\usepackage{amsmath,bm,amssymb}
\usepackage{siunitx}
\usepackage{physics}
\usepackage{lmodern}
\usepackage{tikz}
\usepackage{cutwin}
\usepackage{fancyhdr}
\usepackage{caption}
\usepackage{enumitem}
\usepackage{upgreek}
\usepackage{circuitikz}
\usepackage{mathrsfs}
\usetikzlibrary{snakes,fadings}
\usetikzlibrary{decorations.pathmorphing}
\usetikzlibrary{shapes}
\usetikzlibrary{arrows.meta}

\tikzfading[name=fade left,left color=transparent!30,right color=transparent!70]

\captionsetup{labelformat=empty}
%\renewcommand\thefigure{\theenumi}
\makeatletter
\renewcommand*\maketitle{
    \begin{center}
        \bfseries
        {\Large \@title \par}
        \vskip 1em
        {\global\let\author\@empty}
        {\global\let\date\@empty}
    \end{center}
  \setcounter{footnote}{0}
}
\newcommand{\mlabel}[2]{#2\def\@currentlabel{#2}\label{#1}}
\newcommand{\cube}[5]{
    \pgfmathsetmacro{\cubex}{#2}
    \pgfmathsetmacro{\cubey}{#3}
    \pgfmathsetmacro{\cubez}{#4}
    \filldraw[#5!50,join=round] #1 -- ++(-\cubex,0,0) -- ++(0,-\cubey,0) -- ++(\cubex,0,0) -- cycle;
    \filldraw[#5,join=round] #1 -- ++(0,0,-\cubez) -- ++(0,-\cubey,0) -- ++(0,0,\cubez) -- cycle;
    \filldraw[#5!80,join=round] #1 -- ++(-\cubex,0,0) -- ++(0,0,-\cubez) -- ++(\cubex,0,0) -- cycle;
}
\renewcommand\theenumi{S-\arabic{enumi}}
\renewcommand\labelenumi{\theenumi}
\sisetup{inter-unit-product = \ensuremath { { } \cdot { } } }
\newcommand{\csi}[2]{ \SI{#1}{#2}}
\newcommand*{\dif}{\mathop{}\!\mathrm{d}}
\newcommand{\lbd}[2]{\(\lambda=\csi{#1}{#2}\)}
\newcommand{\ri}[2]{\(n_{#1}={#2}\)}
\newcommand{\tangle}[1]{\(\theta={#1}^\circ\)}
\makeatother

\pagestyle{fancy}
\fancyhf{}
\cfoot{\thepage}
\renewcommand\headrulewidth{0pt}
\title{第15章\ 量子物理}
\author{叶旺全\\大学物理教研室}

\begin{document} 
\maketitle
\begin{enumerate}
    %\item[\mlabel{itm:a}{5-6}] 自定义label引用参考
    \item[15-8] 天狼星的温度大约是\csi{11000}{\celsius}。试由维恩位移定律计算其辐射峰值的波长。
    \item[15-9] 用辐射高温计测得炼钢炉口的辐射出射度为\csi{22.8}{\watt\per\square\cm},试求炉内温度。
    \item[15-12] 钾的截止频率为\csi{4.62e14}{\hertz},今以波长为\csi{435.8}{\nm}的光照射,求钾放出的光电子的初速度。
    \item[15-14] 一个具有\csi{1.0e4}{\eV}能量的光子与一个静止自由电子相碰撞,碰撞后,光子的散射角为\,\ang{60}。试问:(1)光子的波长、频率和能量各改变多少?(2)电子的动能、动量和运动方向又如何?
    \item[15-15] 波长为\csi{0.10}{\nm}的光子入射在碳上,从而产生康普顿效应。从实验中测量到,散射光子的方向与入射光子的方向相垂直。求:(1)散射光子的波长;(2)反冲电子的动能和运动方向。
    \item[15-18] 在氢原子的波尔理论中,当电子由量子数\(n_i=5\)的轨道跃迁到\(n_f=2\)的轨道上时,对外辐射的光的波长为多少?若再将该电子从\(n_f=2\)的轨道跃迁到游离状态,外界需要提供多少能量?
    \item[15-19] 如用能量为\csi{12.6}{\eV}的电子轰击氢原子,将产生哪些谱线?
    \item[15-22] 求动能为\csi{1.0}{\eV}的电子的德布罗意波的波长。
    \item[15-25] 若电子和光子的波长均为\csi{0.20}{\nm},则它们的动量和动能各为多少?
    \item[15-27] 电子位置的不确定量为\csi{5.0e-2}{\nm}时,其速率的不确定量为多少?
    \item[15-29] 一颗质量为\csi{40}{\g}的子弹以\csi{1.0e3}{\m\per\second}的速率飞行,求:(1)其德布罗意波的波长;(2)若测量子弹位置的不确定量为\csi{0.10}{\mm},求其速率的不确定量。
    \item 假设太阳表面温度为\csi{5800}{\kelvin},太阳半径为\csi{6.96e8}{\m}。如果认为太阳的辐射是稳定的,求太阳在一年内由于辐射,它的质量减少了多少?
    \item 铝的逸出功为\csi{4.2}{\eV}。今用波长为\csi{200}{\nm}的紫外光照射到铝表面上,发射的光电子的最大初动能为多少?遏止电势差为多大?铝的红限波长是多大?
    \item 如果一个光子的能量等于一个电子的静止能量,问该光子的频率、波长和动量各为多少?在电磁波谱中属于何种射线?
    \item 在康普顿散射中,入射X射线的波长为\csi{3.0e-3}{\nm},反冲电子的速率为\(0.6c\),求散射光子的波长和散射方向。(能量守恒、动量守恒)
    \item 在基态氢原子被外来单色光激发后发出的巴耳末系中,仅观察到三条谱线,试求:(1) 外来光的波长; (2) 这三条谱线的波长.
    \item 若一个电子的动能等于它的静能,试求该电子的速率和德布罗意波长.
    \item 设粒子在沿\(x\)轴运动时,速率的不确定量为\(\Delta v=\csi{1}{\cm\per\second}\),试估算下列情况下坐标的不确定量\(\Delta x\)。(1) 电子; (2) 质量为\csi{10e-13}{\kg}的布朗粒子;
         (3) 质量为\csi{10e-4}{\kg}的小弹丸.
    \item 氦氖激光器所发出的红光波长为\(\lambda=\csi{632.8}{\nm}\),谱线宽度\(\Delta\lambda=\csi{10e-9}{\nm}\)。试求该光子沿运动方向的位置不确定度 (即波列长度).
\end{enumerate}
\end{document} 