\documentclass[UTF-8]{ctexart} 
\usepackage[a4paper,left=2cm,right=2cm,top=2.5cm,bottom=2.5cm]{geometry}
\usepackage{amsmath,bm,amssymb}
\usepackage{siunitx}
\usepackage{physics}
\usepackage{lmodern}
\usepackage{tikz}
\usepackage{cutwin}
\usepackage{fancyhdr}
\usepackage{caption}
\usepackage{enumitem}
\usepackage{upgreek}
\usepackage{circuitikz}
\usepackage{mathrsfs}
\usetikzlibrary{snakes,fadings}
\usetikzlibrary{decorations.pathmorphing}
\usetikzlibrary{shapes}
\usetikzlibrary{arrows.meta}

\tikzfading[name=fade left,left color=transparent!30,right color=transparent!70]

\captionsetup{labelformat=empty}
%\renewcommand\thefigure{\theenumi}
\makeatletter
\renewcommand*\maketitle{
    \begin{center}
        \bfseries
        {\Large \@title \par}
        \vskip 1em
        {\global\let\author\@empty}
        {\global\let\date\@empty}
    \end{center}
  \setcounter{footnote}{0}
}
\newcommand{\mlabel}[2]{#2\def\@currentlabel{#2}\label{#1}}
\newcommand{\cube}[5]{
    \pgfmathsetmacro{\cubex}{#2}
    \pgfmathsetmacro{\cubey}{#3}
    \pgfmathsetmacro{\cubez}{#4}
    \filldraw[#5!50,join=round] #1 -- ++(-\cubex,0,0) -- ++(0,-\cubey,0) -- ++(\cubex,0,0) -- cycle;
    \filldraw[#5,join=round] #1 -- ++(0,0,-\cubez) -- ++(0,-\cubey,0) -- ++(0,0,\cubez) -- cycle;
    \filldraw[#5!80,join=round] #1 -- ++(-\cubex,0,0) -- ++(0,0,-\cubez) -- ++(\cubex,0,0) -- cycle;
}
\renewcommand\theenumi{S-\arabic{enumi}}
\renewcommand\labelenumi{\theenumi}
\sisetup{inter-unit-product = \ensuremath { { } \cdot { } } }
\newcommand{\csi}[2]{ \SI{#1}{#2}}
\newcommand*{\dif}{\mathop{}\!\mathrm{d}}
\newcommand{\lbd}[2]{\(\lambda=\csi{#1}{#2}\)}
\newcommand{\ri}[2]{\(n_{#1}={#2}\)}
\newcommand{\tangle}[1]{\(\theta={#1}^\circ\)}
\makeatother

\pagestyle{fancy}
\fancyhf{}
\cfoot{\thepage}
\renewcommand\headrulewidth{0pt}
\title{第14章\ 相对论}
\author{叶旺全\\大学物理教研室}

\begin{document} 
\maketitle
\begin{enumerate}
    %\item[\mlabel{itm:a}{5-6}] 自定义label引用参考
    \item[14-6] 设有两个参考系S和S\(^\prime\),它们的原点在\(t=0\)和\(t^\prime=0\)时重合在一起。一个事件在\(S^\prime\)系中发生于\(t^\prime=\csi{8.0e-8}{\second}\),
        \(x^\prime=\csi{60}{\m}\),\(y^\prime=0\),\(z^\prime=0\)处,若S\(^\prime\)系相对于S系以速率\(v=0.60c\)沿\(xx^\prime\)轴运动,问该事件在S系中的时空坐标
        是多少?
    \item[14-7] 一列火车长\csi{0.30}{\km}(火车上观察者测得),以\csi{100}{\km\per\hour}的速度行驶,地面上的观察者发现有两个闪电同时击中火车前后两端。问火车上的观察者测得两闪电
        击中火车前后两端的时间间隔为多少?
    \item[14-10] 设想有一粒子以\(0.050c\)的速率相对实验室参考系运动。此粒子衰变时发射一个电子,电子的速率为\(0.80c\),电子速率的方向与粒子运动方向相同。试求电子相对
        实验室参考系的速度。
    \item[14-12] 以速度\(v\)沿\(x\)方向运动的粒子,在\(y\)方向上发射一个光子,求地面观察者所测得的光子的速度。
    \item[14-14] 设想地球上有一观察者测得一艘宇宙飞船以\(0.60c\)的速率向东飞行,\csi{5.0}{\s}后该飞船将与一个以\(0.80c\)的速率向西飞行的彗星相碰撞。试问:(1)飞船中的人测得
        彗星将以多大的速率向它运动?(2)以飞船中的钟来看,飞船还有多少时间容许它离开航线,以避免与彗星相撞?
    \item[14-15] 在惯性系S中观察到有两个事件发生在同一地点,其时间间隔为\csi{4.0}{\second},从另一个惯性系S\(^\prime\)中观察到这两个事件的时间间隔为\csi{6.0}{\second},试问从S\(^\prime\)系中测量到
        这两个事件的空间间隔是多少?设S\(^\prime\)系以恒定速率相对S系沿\(xx^\prime\)轴运动。
    \item[14-16] 在惯性系S中,有两个事件同时发生在\(xx^\prime\)轴上相距为\csi{1.0e3}{\m}的两处,从惯性系S\(^\prime\)中观测到这两个事件相距为\csi{2.0e3}{\m},试问由S\(^\prime\)系测得这两
        个事件的时间间隔是多少?
    \item[14-19] 一个固有长度为\csi{4.0}{\m}的物体,若以速率\(0.60c\)沿\(x\)轴相对某惯性系运动,试问从该惯性系中来测量,此物体的长度为多少?
    \item[14-23] 若一个电子的总能量\csi{5.0}{\MeV},求该电子的静能、动能、动量和速率。
    \item[14-27] 若把能量\csi{0.50e6}{\eV}给予电子,且电子垂直于磁场运动,则其运动径迹是半径为\csi{2.0}{\cm}的圆。问:(1)该磁场的磁感强度\(B\)有多大?(2)这个电子的动质量为静质量的
        多少倍?
    \item[14-28] 如果将电子由静止加速到速率为\(0.10c\),需对它做多少功?如果将电子的速率由\(0.80c\)加速到\(0.90c\),又需对它做多少功?
    \item 在惯性系S\(^\prime\)中静止的一个圆形轨道,其方程为\({x^\prime}^2+{y^\prime}^2=a^2\),\({z^\prime}^2=0\),试求:在相对S\(^\prime\)系以速度\(v\)运动的惯性系S中观察者将会测得怎样的图像? 
        (另试想,如果一个粒子在S\(^\prime\)中的运动轨迹是圆形,在S系中观察到的运动轨迹会是什么形状?)
    \item 两艘相向飞行的飞船原长都是\(L\),地面观测者测得飞船的长度都为\(L/2\),则两艘飞船相对地面的速度多大? 其中一艘飞船中乘员测得另一艘飞船的长度为多少? 
    \item 假设宇宙飞船从地球出发,沿直线到达月球,距离是\csi{3.84e8}{\m},它的速率在地球上被量得为\(0.30c\)。根据地球上的时钟,这次旅行花多长时间? 根据宇宙飞船所在参考系的测量,
        地球和月球的距离是多少? 怎样根据这个算得的距离,求出宇宙飞船上时钟所读出的旅行时间?
    \item 在S系中观察到两个事件同时发生在\(Ox\)轴上,其间距离是\csi{1}{\m},在S\(^\prime\)系中观察这两个事件之间的空间距离是\csi{2}{\m},求在S\(^\prime\)系中这两个事件的时间间隔。
    \item 在S系中观察到两个事件发生在空间同一地点,第二事件发生在第一事件以后\csi{2}{\second},在另一相对S系运动的S\(^\prime\)系中观察到第二事件是在第一事件\csi{3}{\second}之后发生的,
        求在S\(^\prime\)系中测量两事件之间的位置距离。
    \item 地球上一观察者,看见一飞船A以速度\csi{2.5e8}{\m\per\second}从他身边飞过,另一飞船B以速度\csi{2.0e8}{\m\per\second}跟随A飞行。求:(1) A上的乘客看到B的相对速度; 
        (2) B上的乘客看到A的相对速度。
    \item 某人测得一横置于\(Ox^\prime\)方向的静止棒长为\(l\),质量为\(m_0\),于是求得此棒的线密度为\(m_0/l\)。假定此棒以速度\(v\)在\(Ox^\prime\)方向上运动,此人再测棒的线密度应为多少? 
        若棒在\(Oy^\prime\)方向上运动,它的线密度又为多少?
    \item 在北京正负电子对撞机中,电子可以被加速到\csi{2.8}{\GeV}(\(\csi{1}{\GeV}=\csi{e9}{\eV}\)),(1) 这种电子的速率和光速相比相差多少? (2) 这样的一个电子动量多大? 
        (3) 这种电子在周长为\csi{240}{\m}的储存环内绕行时,它受的向心力多大? 需要多大的偏转磁场 (利用洛伦兹公式\(F=qvB\)进行计算)。    
\end{enumerate}
\end{document} 