\documentclass[UTF-8]{ctexart} 
\usepackage[a4paper,left=2cm,right=2cm,top=2.5cm,bottom=2.5cm]{geometry}
\usepackage{amsmath,bm,amssymb}
\usepackage{siunitx}
\usepackage{physics}
\usepackage{lmodern}
\usepackage{tikz}
\usepackage{cutwin}
\usepackage{fancyhdr}
\usepackage{caption}
\usepackage{enumitem}
\usepackage{upgreek}
\usepackage{circuitikz}
\usepackage{mathrsfs}
\usetikzlibrary{snakes,fadings}
\usetikzlibrary{decorations.pathmorphing}
\usetikzlibrary{shapes}

\tikzfading[name=fade left, left color=transparent!30, right color=transparent!70]

\captionsetup{labelformat=empty}
%\renewcommand\thefigure{\theenumi}
\makeatletter
\renewcommand*\maketitle{
    \begin{center}
        \bfseries
        {\Large \@title \par}
        \vskip 1em
        {\global\let\author\@empty}
        {\global\let\date\@empty}
    \end{center}
  \setcounter{footnote}{0}
}
\newcommand{\mlabel}[2]{#2\def\@currentlabel{#2}\label{#1}}
\newcommand{\cube}[5]{
    \pgfmathsetmacro{\cubex}{#2}
    \pgfmathsetmacro{\cubey}{#3}
    \pgfmathsetmacro{\cubez}{#4}
    \filldraw[#5!50,join=round] #1 -- ++(-\cubex,0,0) -- ++(0,-\cubey,0) -- ++(\cubex,0,0) -- cycle;
    \filldraw[#5,join=round] #1 -- ++(0,0,-\cubez) -- ++(0,-\cubey,0) -- ++(0,0,\cubez) -- cycle;
    \filldraw[#5!80,join=round] #1 -- ++(-\cubex,0,0) -- ++(0,0,-\cubez) -- ++(\cubex,0,0) -- cycle;
}
\renewcommand\theenumi{S-\arabic{enumi}}
\renewcommand\labelenumi{\theenumi}
\sisetup{inter-unit-product = \ensuremath { { } \cdot { } } }
\newcommand{\csi}[2]{ \SI{#1}{#2}}
\newcommand*{\dif}{\mathop{}\!\mathrm{d}}
\makeatother

\pagestyle{fancy}
\fancyhf{}
\cfoot{\thepage}
\renewcommand\headrulewidth{0pt}
\title{第8章\ 电磁感应 电磁场}
\author{叶旺全\\大学物理教研室}

\begin{document} 
\maketitle
\begin{enumerate}
    %\item[\mlabel{itm:a}{5-6}] 自定义label引用参考
    \item[\mlabel{itm:6}{8-6}] 一个铁芯上绕有线圈100匝,已知铁芯中磁通量与时间的关系为\(\varPhi=\csi{8.0e-5}{}\sin100\uppi t\),
        式中\(\varPhi\)的单位为\unit{\weber},\(t\)的单位为\unit{\s}。求在\(t=\csi{1.0e-2}{\s}\)时,线圈中的感应电动势。
    \item[\mlabel{itm:7}{8-7}] 两根相距为\(d\)的无限长平行直导线,通以大小相等流向相反的电流,且电流均以\(\frac{\dif I}{\dif t}\)的变化率增长。
        若有一边长为的的正方形线圈与两导线处于同一平面内,如图所示,求线圈的感应电动势。
    \item[\mlabel{itm:10}{8-10}] 如图所示,一根长直导线中通有\(I=\csi{5.0}{\A}\)的电流,在距导线\csi{9.0}{\cm}处,放一个面积为
        \csi{0.10}{\square\cm}、10匝的小圆线圈,线圈中的磁场可看作是均匀的。今在\csi{1.0e-2}{\s}内把此线圈移至距长直导线\csi{10.0}{\cm}处。
        (1)求线圈中的平均感应电动势;(2)设线圈的电阻为\csi{1.0e-2}{\ohm},求通过线圈横截面的感应电荷。
    \item[\mlabel{itm:13}{8-13}] 如图所示,长为\(L\)的导体棒\(OP\)处于均匀磁场中,并绕\(OO^\prime\)轴以角速度\(\omega\)旋转,
        棒与转轴见的夹角恒为\(\theta\),磁感强度\(\bm{B}\)与转轴平行,求OP棒在图示位置处的电动势。
\end{enumerate}
\end{document} 